%
\documentclass[12pt]{exam2}
%\printanswers

\usepackage{amsmath}

\begin{document}

%\maketitle
\begin{center}
{\LARGE Quiz 2 - Basics of MLE}\\\vspace{2mm}
\vspace{3mm}
{\large Advanced Methods}\\\vspace{2mm}
%Carlisle Rainey\symbolfootnote[1]{Carlisle Rainey is an Assistant Professor of Political Science, University at Buffalo, SUNY, 520 Park Hall, Buffalo, NY 14260 (\href{mailto:rcrainey@buffalo.edu}{rcrainey@buffalo.edu}).}
\end{center}

\begin{questions}
\question The homework asks you to find the MLE of the mean $\mu$ for the probability model $y ~ N(\mu, 1)$--that is, when the variance is \textit{known} and equal to one. 
\begin{parts}
\part Find the MLE for the mean when the variance is \textit{unknown}, but fixed at $\sigma^2$. 
\part How does having an unknown variance parameter impact the MLE for the mean? 
\end{parts}
Hint: The normal pdf is given by
\begin{equation*}
f(y_i | \mu, \sigma) = \dfrac{1}{\sigma \sqrt{2\pi}}e^{-\dfrac{(y_i - \mu)^2}{2\sigma^2}}.
\end{equation*}

\end{questions}
\end{document}





