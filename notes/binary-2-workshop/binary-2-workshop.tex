%
\documentclass[12pt]{article}

% The usual packages
\usepackage{fullpage}
\usepackage{breakcites}
\usepackage{setspace}
\usepackage{endnotes}
\usepackage{float}
\usepackage{amsmath}
\usepackage{amsfonts}
\usepackage{amssymb}
\usepackage{rotating}
\usepackage{dcolumn}
\usepackage{longtable}
\usepackage{microtype}
\usepackage{graphicx}
\usepackage{hyperref}
%\usepackage[usenames,dvipsnames]{color}
\usepackage{url}
\usepackage{natbib}
\usepackage{framed}
\usepackage{epigraph}
\usepackage{lipsum}
\usepackage[font=small,labelfont=sc]{caption}
\restylefloat{table}
\bibpunct{(}{)}{;}{a}{}{,}

% Set paragraph spacing the way I like
\parskip=0pt
\parindent=20pt

% Define mathematical results
\newtheorem{lemma}{Lemma}
\newtheorem{proposition}{Proposition}
\newtheorem{theorem}{Theorem}
\newtheorem{claim}{Claim}
\newenvironment{proof}[1][Proof]{\begin{trivlist}
\item[\hskip \labelsep {\bfseries #1}]}{\end{trivlist}}
\newenvironment{definition}[1][Definition]{\begin{trivlist}
\item[\hskip \labelsep {\bfseries #1}]}{\end{trivlist}}
\newenvironment{example}[1][Example]{\begin{trivlist}
\item[\hskip \labelsep {\bfseries #1}]}{\end{trivlist}}
\newenvironment{remark}[1][Remark]{\begin{trivlist}
\item[\hskip \labelsep {\bfseries #1}]}{\end{trivlist}}

% Set up fonts the way I like
\usepackage{tgpagella}
\usepackage[T1]{fontenc}
\usepackage[bitstream-charter]{mathdesign}

%% Set up lists the way I like
% Redefine the first level
\renewcommand{\theenumi}{\arabic{enumi}.}
\renewcommand{\labelenumi}{\theenumi}
% Redefine the second level
\renewcommand{\theenumii}{\alph{enumii}.}
\renewcommand{\labelenumii}{\theenumii}
% Redefine the third level
\renewcommand{\theenumiii}{\roman{enumiii}.}
\renewcommand{\labelenumiii}{\theenumiii}
% Redefine the fourth level
\renewcommand{\theenumiv}{\Alph{enumiv}.}
\renewcommand{\labelenumiv}{\theenumiv}
% Eliminate spacing around lists
\usepackage{enumitem}
\setlist{nolistsep}

% Create footnote command so that my name
% has an asterisk rather than a one.
\long\def\symbolfootnote[#1]#2{\begingroup%
\def\thefootnote{\fnsymbol{footnote}}\footnote[#1]{#2}\endgroup}

% Create the colors I want
\usepackage{color}
\definecolor{darkred}{RGB}{100,0,0}

% enable comments in pdf
\newcommand{\kelly}[1]{\textcolor{blue}{#1}}
\newcommand{\carlisle}[1]{\textcolor{magenta}{#1}}


\begin{document}

\begin{center}
{\LARGE Workshop on Modeling Binary Outcomes}\\\vspace{2mm}
Carlisle Rainey\symbolfootnote[3]{Carlisle Rainey is Assistant Professor of Political Science, University at Buffalo, SUNY, 520 Park Hall, Buffalo, NY 14260 (\href{mailto:rcrainey@buffalo.edu}{rcrainey@buffalo.edu}).}
\end{center}
\vspace{2mm}

Statistical modeling requires combining substantive knowledge with theoretical (of the statistical sort) and computation work. You should be getting better at mixing these together. I don't want my classes to encourage rote behavior. Instead, I want to encourage creativity. To that end, we're going to work through a full problem in class today.

We have a generic framework for going from a substantive theoretical claims to substantive empirical results.
\begin{enumerate}
\item Write down a probability model of the form $y_i \sim f(\theta_i)$, where $\theta = g^{-1}(X_i\beta)$.
	\begin{enumerate}
	\item We choose $f$ so that the support of $f$ matches the possible values of the data and so that $f$ approximately captures the distribution of the data given $X_i$. (Because we've worked only with binary outcomes, we haven't seen the latter in action, but we'll get there.)
	\item We choose $g$ to map $\Theta$ to $\mathbb{R}$ and capture any desired substantive relationships. For example, we can motivate compression in logistic regression from a substantive perspective. However, it is important that our models be able to represent our proposed hypotheses and the plausible alternatives.
	\end{enumerate}
\item We use maximum likelihood to estimate the model coefficients and their standard errors.
	\begin{enumerate}
	\item Write down the probability of the data given the parameters $p(y | \beta) = L(\beta | y)$.
	\item Take the log of the likelihood and simplify.
	\item Find the vector $\hat{\beta}$ that maximizes the $\log L$. This is the MLE.
	\item Estimate the covariance matrix $\hat{\Sigma}_{\hat{\beta}}$ by taking the inverse of the negative of the Hessian matrix.
	\end{enumerate}
\item Choose and motivate a quantity of interest $q$ (a function of the parameters and hypothetical row vectors; e.g., risk-ratio, first-difference, marginal effects, second-differences, differences in marginal effects, etc). Find the MLE of the $q$ and obtain the confidence interval.
	\begin{enumerate}
	\item Use the invariance principle to find $\hat{q}$.
	\item Take $J$ simulations from $q$'s ``posterior'' indirectly by simulating the vector $\tilde{\beta}$ $J$ times from it's ``posterior,'' which is approximately $N(\hat{\beta}, \hat{\Sigma}_{\hat{\beta}})$ and plugging each simulated $\tilde{\beta}_j$ into the function $q$.
	\item Take the 5th and 95th percentiles of these simulation to obtain a 90\% confidence interval, etc.
	\end{enumerate}
\end{enumerate}

\section*{Problem 1}

\subsection*{Write down the probability model.}

\begin{enumerate}
\item Let's examine the logit function $\Lambda(x) = \frac{1}{1 + e^{-x}}$. What point $x_0$ has the largest marginal effect? 
\item Suppose that $X_c\beta = x_0$ in the usual logit model. What is the probability of an event?
\item I'm going to propose a different inverse link function $g^{-1}(x) = 1 - (1 + e^x)^{-\alpha}$. What $x_0$ maximizes this function?
\item Might there be a substantive reason to choose $g^{-1}$ over $\Lambda$?
\item Write down a probability model that uses $g^{-1}$.
\end{enumerate}

\subsection*{Estimate the parameters and standard errors.}

\begin{enumerate}
\item Derive the log-likelihood for our new model.
\item Write a function in R that calculates this log-likelihood. It's convenient to write a function for our new $g^{-1}$ as well.
\item Test it and make sure it works.
\item Write a function in R that optimizes the log-likelihood and returns the coefficient estimates, their standard errors and the covariance matrix.
\item Test is and make sure it works.
\item Try it out on the data below using some reasonable explanatory variables.
\end{enumerate}

\begin{verbatim}
# load packages
library(foreign)

# load data
d <- read.dta("http://crain.co/am-files/data/bde.dta")

# drop the -1's (I'm not sure what they mean)
d <- d[d$newvote != -1, ]
\end{verbatim}

\subsection*{Calculate the quantities of interest and confidence interval.}

\begin{enumerate}
\item Choose a quantity of interest. Explain why you chose it.
\item Use the invariance principle to obtain $\hat{q}$.
\item Simulate $J \geq 1000$ model coefficients $\hat{\beta}^{[j]}$ for $j \in 1,..., J$ from the ``posterior.''
\item Plug these coefficients in to obtain $\tilde{q}^{[j]}$ and summarize $\tilde{q}$ as needed.
\end{enumerate}

\section*{Problem 2}

\subsection*{Write down the probability model.}

\begin{enumerate}
\item Suppose that we have two unobserved (and possibly unobservable) binary outcomes $z_i^X \sim \text{Bernoulli}(\pi^{[X]}_i)$, where $\pi^{[X]}_i = \text{logit}^{-1}(X_i\beta)$, and $z_i^W \sim \text{Bernoulli}(\pi^{[W]}_i)$, where $\pi^{[w]}_i = \text{logit}^{-1}(W_i\gamma)$. Now suppose that we can observe a binary outcome $y_i$ that equals one if $z_i^{[X]} = 1$ \underline{and} $z_i^{[W]} = 1$ and zero otherwise. We'll suppose that $z_i^{[X]}$ and $z_i^{[W]}$ because that makes things a little easier. Write down the probability $\text{Pr}(y_i = 1)$.
\end{enumerate}

\subsection*{Estimate the parameters and standard errors.}

\begin{enumerate}
\item Derive the log-likelihood.
\item Write a function in R that calculates this log-likelihood. It may be convenient for write a function to calculate $\text{Pr}(y_i = 1)$.
\item Test it and make sure it works.
\item Write a function in R that optimizes the log-likelihood and returns the coefficient estimates, their standard errors and the covariance matrix.
\item Test is and make sure it works.
\item Try it out on the data below using some reasonable explanatory variables. We might think that the two unobservabilty probabilities are $\Pr(\text{Vote}~|~\text{Register})$ and $\Pr(\text{Register})$. You'll need to choose what variables belong in $X$, $W$, both, or neither.
\end{enumerate}

\begin{verbatim}
# load packages
library(foreign)

# load data
d <- read.dta("http://crain.co/am-files/data/bde.dta")

# drop the -1's (I'm not sure what they mean)
d <- d[d$newvote != -1, ]
\end{verbatim}

\subsection*{Calculate the quantities of interest and confidence interval.}

\begin{enumerate}
\item Choose a quantity of interest. Explain why you chose it.
\item Use the invariance principle to obtain $\hat{q}$.
\item Simulate $J \geq 1000$ model coefficients $\text{vec}(\tilde{\beta}, \tilde{\gamma})^{[j]}$ for $j \in 1,..., J$ from the ``posterior.''
\item Plug these coefficients in to obtain $\tilde{q}^{[j]}$ and summarize $\tilde{q}$ as needed.
\end{enumerate}


\end{document}